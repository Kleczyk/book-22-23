\documentclass[../main.tex]{subfiles}



\begin{document}

\chapter{}
\label{cha:cha_7}


\section{}
\begin{lstlisting}[numbers=none]
>> Aug = [A eye(size(A))]
\end{lstlisting}
\bigbreak
Here’s an example session of how it can be employed.
\bigbreak
\begin{lstlisting}[numbers=none]
>> A = rand(3)

A =
	 0.9501 0.4860 0.4565
	 0.2311 0.8913 0.0185
 	 0.6068 0.7621 0.8214
 
>> Aug = [A eye(size(A))]

Aug =
 	 0.9501 0.4860 0.4565 1.0000 0 0
 	 0.2311 0.8913 0.0185 0 1.0000 0
   0.6068 0.7621 0.8214 0 0 1.0000 
\end{lstlisting}
\bigbreak


\section{}
\begin{enumerate}[label=\bfseries(\alph*)]
\item $\begin{array}{lllll}[A]: 3 \times 2 & {[B]: 3 \times 3} & \{C\}: 3 \times 1 & \text { [D]: } 2 \times 4\\ &\hspace*{-1.75cm} [E]: 3 \times 3 &\hspace*{-1.75cm} [F]: 2 \times 3 &\hspace*{-1.75cm} \lfloor G\rfloor: 1 \times 3  \\\end{array} $
\bigbreak
\item square: $[B],[E]$; column: $\{C\}$, row: $\lfloor G\rfloor$
\bigbreak
\item $a_{12}=5, b_{23}=6, d_{32}=$ undefined, $e_{22}=1, f_{12}=0, g_{12}=6$
\bigbreak
\item 
MATLAB can be used to perform the operations
\bigbreak
(1) $[E]+[B]=\left[\begin{array}{ccc}5 & 8 & 13 \\ 8 & 3 & 9 \\ 5 & 0 & 9\end{array}\right]$\quad\quad\quad\quad
(2) $[E]-[B]=\left[\begin{array}{ccc}3 & -2 & 1 \\ -6 & 1 & 3 \\ -3 & 0 & -1\end{array}\right]$\\\\\\
(3) $[A]+[F]=$ undefined\quad\quad\quad\quad\quad\quad\quad
(4) $5[F]=\left[\begin{array}{ccc}20 & 15 & 35 \\ 5 & 10 & 30 \\ 5 & 0 & 20\end{array}\right]$\\\\\\
(5) $[A] \times[B]=$ undefined\quad\quad\quad\quad\quad\quad\quad
(6) $[B] \times[A]=\left[\begin{array}{ll}54 & 68 \\ 36 & 45 \\ 24 & 29\end{array}\right]$\\\\\\\\
(7) $[G] \times[C]=56$\quad\quad\quad\quad\quad\quad\quad\quad\quad\quad
(8) $[C]^{T}=\left[\begin{array}{lll}2 & 6 & 1\end{array}\right\rfloor$\\\\\\
(9) $[D]^{T}=\left[\begin{array}{ll}5 & 2 \\ 4 & 1 \\ 3 & 7 \\ 6 & 5\end{array}\right]$\quad\quad\quad\quad\quad\quad\quad\quad
(10) $I \times[B]=\left[\begin{array}{lll}4 & 3 & 7 \\ 1 & 2 & 6 \\ 1 & 0 & 4\end{array}\right]$
\bigbreak
\section{}
The terms can be collected to give
\bigbreak
$\left[\begin{array}{ccc}
-7 & 3 & 0 \\
0 & 4 & 7 \\
-4 & 3 & -7
\end{array}\right]\left\{\begin{array}{l}
x_{1} \\
x_{2} \\
x_{3}
\end{array}\right\}=\left\{\begin{array}{c}
10 \\
-30 \\
40
\end{array}\right\}$
\bigbreak
Here is the MATLAB session:
\bigbreak
\begin{lstlisting}[numbers=none]
>> A = [-7 3 0;0 4 7;-4 3 -7];
>> b = [10;-30;40];
>> x = A\b

x =
 	 -1.0811
 	  0.8108
 	 -4.7490
 	
>> AT = A'

AT =
	 -7 0 -4
 	  3 4 3
 	  0 7 -7
 	 
>> AI = inv(A)

AI =
 	 -0.1892 0.0811 0.0811
 	 -0.1081 0.1892 0.1892
 	  0.0618 0.0347 -0.1081
\end{lstlisting}
\bigbreak

\section{}
$
\begin{aligned}
&{[X] \times[Y]=\left[\begin{array}{cc}
23 & -8 \\
55 & 56 \\
-17 & 24
\end{array}\right]} \\\\
&{[X] \times[Z]=\left[\begin{array}{cc}
12 & 8 \\
-30 & 52 \\
-23 & 2
\end{array}\right]} \\\\
&{[Y] \times[Z]=\left[\begin{array}{cc}
4 & 8 \\
-47 & 34
\end{array}\right]} \\\\
&{[Z] \times[Y]=\left[\begin{array}{cc}
6 & 16 \\
-20 & 32
\end{array}\right]}
\end{aligned}$
\bigbreak


\section{}
Terms can be combined to yield
\bigbreak
$\hspace*{0.1cm}2 k x_{1}-k x_{2} \hspace*{1cm}=m_{1} g$ \\
$
\begin{aligned}
-k x_{1}+2 k x_{2}-k x_{3} &=m_{2} g \\
-k x_{2}+k x_{3} &=m_{3} g
\end{aligned}$
\bigbreak
Substituting the parameter values
\bigbreak
$
\left[\begin{array}{ccc}
20 & -10 & 0 \\
-10 & 20 & -10 \\
0 & -10 & 10
\end{array}\right]\left\{\begin{array}{l}
x_{1} \\
x_{2} \\
x_{3}
\end{array}\right\}=\left\{\begin{array}{c}
19.62 \\
29.43 \\
24.525
\end{array}\right\}
$
\bigbreak
A MATLAB session can be used to obtain the solution for the displacements
\bigbreak
\begin{lstlisting}[numbers=none]
>> K=[20 -10 0;-10 20 -10;0 -10 10];
>> m=[2;3;2.5];
>> mg=m*9.81;
>> x=K\mg

x =
 	 7.3575
 	 12.7530
 	 15.2055 
\end{lstlisting}
\bigbreak


\section{}
The mass balances can be written as
\bigbreak %{llrll}
$\begin{array}{llllr}
\left(Q_{15}+Q_{12}\right) c_{1} && -Q_{31} c_{3} &&=Q_{01} c_{01} \\\\
-Q_{12} c_{1}+\left(Q_{23}+Q_{24}+Q_{25}\right) c_{2} &&&& =0 \\\\
&-Q_{23} c_{2}+\left(Q_{31}+Q_{34}\right) c_{3} &&& =Q_{03} c_{03} \\\\
&-Q_{24} c_{2} & -Q_{34} c_{3}+Q_{44} c_{4} & -Q_{54} c_{5} & =0 \\\\
-Q_{15} c_{1} & -Q_{25} c_{2} && +\left(Q_{54}+Q_{55}\right) c_{5} & =0
\end{array}$
\bigbreak
The parameters can be substituted and the result written in matrix form as
\bigbreak
$\left[\begin{array}{ccccc}
6 & 0 & -1 & 0 & 0 \\
-3 & 3 & 0 & 0 & 0 \\
0 & -1 & 9 & 0 & 0 \\
0 & -1 & -8 & 11 & -2 \\
-3 & -1 & 0 & 0 & 4
\end{array}\right]\left\{\begin{array}{l}
c_{1} \\
c_{2} \\
c_{3} \\
c_{4} \\
c_{5}
\end{array}\right\}=\left\{\begin{array}{c}
50 \\
0 \\
160 \\
0 \\
0
\end{array}\right\}$
\bigbreak
MATLAB can then be used to solve for the concentrations
\bigbreak
\begin{lstlisting}[numbers=none]
>> Q = [6 0 -1 0 0;
-3 3 0 0 0;
0 -1 9 0 0;
0 -1 -8 11 -2;
-3 -1 0 0 4];
>> Qc = [50;0;160;0;0];
>> c = Q\Qc

c =
 	 11.5094
	 11.5094
 	 19.0566
	 16.9983
 	 11.5094
\end{lstlisting}
\bigbreak

\section{}
The problem can be written in matrix form as
\bigbreak
$\left[\begin{array}{cccccc}
0.866 & 0 & -0.5 & 0 & 0 & 0 \\
0.5 & 0 & 0.866 & 0 & 0 & 0 \\
-0.866 & -1 & 0 & -1 & 0 & 0 \\
-0.5 & 0 & 0 & 0 & -1 & 0 \\
0 & 1 & 0.5 & 0 & 0 & 0 \\
0 & 0 & -0.866 & 0 & 0 & -1
\end{array}\right]\left\{\begin{array}{c}
F_{1} \\
F_{2} \\
F_{3} \\
H_{2} \\
V_{2} \\
V_{3}
\end{array}\right\}=\left\{\begin{array}{c}
0 \\
-1000 \\
0 \\
0 \\
0 \\
0
\end{array}\right\}$
\bigbreak
MATLAB can then be used to solve for the forces and reactions,
\begin{lstlisting}[numbers=none]
>> A = [0.866 0 -0.5 0 0 0;
0.5 0 0.866 0 0 0;
-0.866 -1 0 -1 0 0;
-0.5 0 0 0 -1 0;
0 1 0.5 0 0 0;
0 0 -0.866 0 0 -1]
>> b = [0 -1000 0 0 0 0]';
>> F = A\b

F =
 -500.0220
  433.0191
 -866.0381
 -0.0000
  250.0110
  749.9890
\end{lstlisting}
\bigbreak
Therefore,
\bigbreak
$\begin{array}{llllll}
F_{1}=-500 && F_{2}=433 && F_{3}=-866 \\
H_{2}=0 && V_{2}=250 && V_{3}=750 
\end{array}$
\bigbreak


\section{}
\bigbreak
The problem can be written in matrix form as
\bigbreak
$\left[\begin{array}{cccccc}1 & 1 & 1 & 0 & 0 & 0 \\ 0 & -1 & 0 & 1 & -1 & 0 \\ 0 & 0 & -1 & 0 & 0 & 1 \\ 0 & 0 & 0 & 0 & 1 & -1 \\ 0 & 10 & -10 & 0 & -15 & -5 \\ 5 & -10 & 0 & -20 & 0 & 0\end{array}\right]\left\{\begin{array}{l}i_{12} \\ i_{52} \\ i_{32} \\ i_{65} \\ i_{54} \\ i_{43}\end{array}\right\}=\left\{\begin{array}{c}0 \\ 0 \\ 0 \\ 0 \\ 0 \\ 200\end{array}\right\}$
\bigbreak
MATLAB can then be used to solve for the currents,
\bigbreak
\begin{lstlisting}[numbers=none]
>> A = [1 1 1 0 0 0 ;
0 -1 0 1 -1 0;
0 0 -1 0 0 1;
0 0 0 0 1 -1;
0 10 -10 0 -15 -5;
5 -10 0 -20 0 0];
>> b = [0 0 0 0 0 200]';
>> i = A\b

i =
 	  6.1538
	 -4.6154
 	 -1.5385
 	 -6.1538
 	 -1.5385
 	 -1.5385
\end{lstlisting}
\bigbreak

\section{}
\begin{lstlisting}[numbers=none]
 >> k1 = 10;k2 = 40;k3 = 40;k4 = 10;
>> m1 = 1;m2 = 1;m3 = 1;
>> km = [(1/m1)*(k2+k1), -(k2/m1),0;
-(k2/m2), (1/m2)*(k2+k3), -(k3/m2);
0, -(k3/m3),(1/m3)*(k3+k4)];
>> x = [0.05;0.04;0.03];
>> kmx = km*x

kmx =
 		0.9000
		0.0000
 	 -0.1000
 
\end{lstlisting}
\bigbreak
Therefore, $\ddot{x}_{1}=-0.9, \ddot{x}_{2}=0$, and $\ddot{x}_{3}=0.1 \mathrm{~m} / \mathrm{s}^{2}$.

 
 
\end{enumerate}
\end{document}

