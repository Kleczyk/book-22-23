\documentclass[../main.tex]{subfiles}






\begin{document}
\chapter{}
\label{cha:cha_17}

\section{}
The integral can be evaluated analytically as,
	\bigbreak
$\displaystyle I=\int_{1}^{2}\left(2 x+\dfrac{3}{x}\right)^{2} d x=\int_{1}^{2} 4 x^{2}+12+9 x^{-2} d x $
	\bigbreak
$\displaystyle I=\left[\dfrac{4 x^{3}}{3}+12 x-\dfrac{9}{x}\right]_{1}^{2}=\dfrac{4(2)^{3}}{3}+12(2)-\dfrac{9}{2}-\dfrac{4(1)^{3}}{3}-12(1)+\dfrac{9}{1}=25.8333$
	\bigbreak
The tableau depicting the implementation of Romberg integration to $\varepsilon_{\mathrm{s}}=0.5 \%$ is
	\bigbreak
\begin{tabular}{lccc}
iteration $\rightarrow$ & 1 & 2 & 3 \\
$\varepsilon_{t} \rightarrow$ & $6.9355 \%$ & $0.1613 \%$ & $0.0048 \%$ \\
$\varepsilon_{a} \rightarrow$ &  & $1.6908 \%$ & $0.0098 \%$ \\
\multicolumn{1}{c}{1} & $27.62500000$ & $25.87500000$ & $25.83456463$ \\
\multicolumn{1}{c}{2} & $26.31250000$ & $25.83709184$ &  \\
\multicolumn{1}{c}{4} & $25.95594388$ &  &  \\
\end{tabular}
	\bigbreak
Thus, the result is $25.83456$.
	\bigbreak



\section{}
\begin{enumerate}[label=\bfseries(\alph*)]
\item The integral can be evaluated analytically as,
	\bigbreak
$\displaystyle I=\left[-0.01094 x^{5}+0.21615 x^{4}-1.3854 x^{3}+3.14585 x^{2}+2 x\right]_{0}^{8}=34.87808$
	\bigbreak
\item The tableau depicting the implementation of Romberg integration to $\varepsilon_{\mathrm{s}}=0.5 \%$ is
	\bigbreak
\begin{tabular}{lrrrr}
iteration $\rightarrow$ & 1 & 2 & 3 & 4\\
$\varepsilon_{t} \rightarrow$ & $20.1699\%$ & $42.8256\%$ & $0.0000\%$ & $0.0000\%$\\
$\varepsilon_{a} \rightarrow$ &  & $9.9064\%$ & $2.6766\%$ & $0.000000\%$\\
\multicolumn{1}{c}{1}&27.84320000&19.94133333&34.87808000&34.87808000\\
\multicolumn{1}{c}{2}&21.91680000&33.94453333&34.87808000\\
\multicolumn{1}{c}{4}&30.93760000&34.81973333\\
\multicolumn{1}{c}{8}&33.84920000&\\
\end{tabular}
	\bigbreak
Thus, the result is exact.
	\bigbreak
\item The transformations can be computed as
	\bigbreak
$x=\dfrac{(8+0)+(8-0) x_{d}}{2}=4+4 x_{d} \quad d x=\dfrac{8-0}{2} d x_{d}=4 d x_{d}$
	\bigbreak
These can be substituted to yield
	\bigbreak
$\displaystyle I=\int_{-1}^{1}\left[-0.0547\left(4+4 x_{d}\right)^{4}+0.8646\left(4+4 x_{d}\right)^{3}-4.1562\left(4+4 x_{d}\right)^{2}+6.2917\left(4+4 x_{d}\right)+2\right] 4 d x_{d}$
	\bigbreak
The transformed function can be evaluated using the values from Table $17.1$
	\bigbreak
$I=0.5555556 f(-0.774596669)+0.8888889 f(0)+0.5555556 f(0.774596669)=34.87808$
	\bigbreak
which is exact.
	\bigbreak
\item
\begin{lstlisting}[numbers=none]
>> format long
>> y = inline('-0.0547*x.^4+0.8646*x.^3-4.1562*x.^2+6.2917*x+2');
>> I = quad(y,0,8)


I =
	34.87808000000000
\end{lstlisting}
\end{enumerate}
	\bigbreak



\section{}
 Although it's not required, the analytical solution can be evaluated simply as
	\bigbreak
$\displaystyle I=\int_{0}^{3} x e^{x} d x=\left[e^{x}(x-1)\right]_{0}^{3}=41.17107385$
	\bigbreak
\begin{enumerate}[label=\bfseries(\alph*)]
\item The tableau depicting the implementation of Romberg integration to $\varepsilon_{\mathrm{s}}=0.5 \%$ is
	\bigbreak
\begin{tabular}{lrrr}
iteration $\rightarrow$ & 1 & 2 & 3 \\
$\varepsilon_{t} \rightarrow$ & $119.5350 \%$ & $5.8349 \%$ & $0.1020 \%$ \\
$\varepsilon_{a} \rightarrow$ &  & $26.8579 \%$ & $0.3579 \%$ \\
\multicolumn{1}{c}{1} & $90.38491615$ & $43.57337260$ & $41.21305531$ \\
\multicolumn{1}{c}{2} & $55.27625849$ & $41.36057514$ &  \\
\multicolumn{1}{c}{4} & $44.83949598$ &  &  \\
\end{tabular}
	\bigbreak
which represents a percent relative error of $0.102 \%$.
	\bigbreak
\item The transformations can be computed as
	\bigbreak
$x=\dfrac{(3+0)+(3-0) x_{d}}{2}=1.5+1.5 x_{d} \quad d x=\dfrac{3-0}{2} d x_{d}=1.5 d x_{d}$
	\bigbreak
These can be substituted to yield
	\bigbreak
$\displaystyle I=\int_{-1}^{1}\left[\left(1.5+1.5 x_{d}\right) e^{1.5+1.5 x_{d}}\right] 1.5 d x_{d}$
	\bigbreak
The transformed function can be evaluated using the values from Table $17.1$
	\bigbreak
$I=f(-0.577350269)+f(0.577350269)=39.6075058$
	\bigbreak
which represents a percent relative error of $3.8 \%$.
	\bigbreak
\item Using MATLAB
	\bigbreak
\begin{lstlisting}[numbers=none]
>> format long
>> I = quad(inline('x.*exp(x)'),0,3) 

I =
	41.17107385090233
\end{lstlisting}
	\bigbreak
which represents a percent relative error of $1.1 \times 10^{-8} \%$.
	\bigbreak
\begin{lstlisting}[numbers=none]
>> I = quadl(inline('x.*exp(x)'),0,3)


I =
	41.17107466800178
\end{lstlisting}
	\bigbreak
which represents a percent relative error of $2 \times 10^{-6} \%$.
	\bigbreak
\end{enumerate}



\section{}
The exact solution can be evaluated simply as
	\bigbreak
\begin{lstlisting}[numbers=none]
>> format long
>> erf(1.5)


ans =
	0.96610514647531 
\end{lstlisting}
\begin{enumerate}[label=\bfseries(\alph*)]
\item The transformations can be computed as
	\bigbreak
$x=\dfrac{(1.5+0)+(1.5-0) x_{d}}{2}=0.75+0.75 x_{d} \quad d x=\dfrac{1.5-0}{2} d x_{d}=0.75 d x_{d}$
	\bigbreak
These can be substituted to yield
	\bigbreak
$\displaystyle I=\dfrac{2}{\sqrt{\pi}} \int_{-1}^{1}\left[e^{-\left(0.75+0.75 x_{d}\right)^{2}}\right] 0.75 d x_{d}$
	\bigbreak
The transformed function can be evaluated using the values from Table $17.1$
	\bigbreak
$I=f(-0.577350269)+f(0.577350269)=0.974173129$
	\bigbreak
which represents a percent relative error of $0.835 \%$.
	\bigbreak
\item The transformed function can be evaluated using the values from Table 17.1
	\bigbreak
$I=0.5555556 f(-0.774596669)+0.8888889 f(0)+0.5555556 f(0.774596669)=0.965502083$
	\bigbreak
which represents a percent relative error of $0.062 \%$.
	\bigbreak
\end{enumerate}



\section{}
\begin{enumerate}[label=\bfseries(\alph*)]
\item The tableau depicting the implementation of Romberg integration to $\varepsilon_{\mathrm{s}}=0.5 \%$ is
	\bigbreak
$
\begin{tabular}{lrrrr}
iteration $\rightarrow$ & 1 & 2 & 3 & 4 \\
 $\varepsilon_{a} \rightarrow$ & & 19.1131 $\%$ & 1.0922 $\%$ & 0.035826 $\%$ \\
1 & 199.66621287 & 847.93212300 & 1027.49455856 & 1051.60670352 \\
2 & 685.86564547 & 1016.27190634 & 1051.22995126 & \\ 4 & 933.67034112 & 1049.04507345 & &
\\ 8 & 1020.20139037 & & &
\end{tabular}
$
	\bigbreak
Note that if 8 iterations are implemented, the method converges on a value of 1053.38523686. This result is also obtained if you use the composite Simpson's $1 / 3$ rule with 1024 segments.
	\bigbreak
\item The transformations can be computed as
	\bigbreak
$x=\dfrac{(30+0)+(30-0) x_{d}}{2}=15+15 x_{d} \quad d x=\dfrac{30-0}{2} d x_{d}=15 d x_{d}$
	\bigbreak
These can be substituted to yield
	\bigbreak
$\displaystyle I=200 \int_{-1}^{1}\left[\dfrac{15+15 x_{d}}{22+15 x_{d}} e^{-2.5\left(15+15 x_{d}\right) / 30}\right] 15 d x_{d}$
	\bigbreak
The transformed function can be evaluated using the values from Table $17.1$
	\bigbreak
$I=f(-0.577350269)+f(0.577350269)=1162.93396$
	\bigbreak
\item Interestingly, the quad function encounters a problem and exceeds the maximum number of iterations
	\bigbreak
\begin{lstlisting}[numbers=none]
>> format long
>> I = quad(inline('200*x/(7+x)*exp(-2.5*x/30)'),0,30)
Warning: Maximum function count exceeded; singularity likely.
(Type "warning off MATLAB:quad:MaxFcnCount" to suppress this
warning.)
> In quad at 88


I =
	1.085280043451920e+003 
\end{lstlisting}
	\bigbreak
The quadl function converges rapidly, but does not yield a very accurate result:
	\bigbreak
\begin{lstlisting}[numbers=none]
>> I = quadl(inline('200*x/(7+x)*exp(-2.5*x/30)'),0,30)


I =
	1.055900924411335e+003 
\end{lstlisting}
\end{enumerate}
	\bigbreak



\section{}
The integral to be evaluated is
	\bigbreak
$\displaystyle I=\int_{0}^{1 / 2}\left(10 e^{-t} \sin 2 \pi t\right)^{2} d t$
	\bigbreak
\begin{enumerate}[label=\bfseries(\alph*)]
\item The tableau depicting the implementation of Romberg integration to $\varepsilon_{\mathrm{s}}=0.1 \%$ is
	\bigbreak
\begin{tabular}{lrrrr}
\multicolumn{1}{c}{iteration $\rightarrow$} & 1 & 2 & 3 & 4 \\
$\varepsilon_{a} \rightarrow$ &  & $25.0000 \%$ & $2.0824 \%$ & $0.025340 \%$ \\
\multicolumn{1}{c}{1} & $0.00000000$ & $20.21768866$ & $15.16502516$ & $15.41501768$ \\
\multicolumn{1}{c}{2} & $15.16326649$ & $15.48081663$ & $15.41111155$ &  \\
\multicolumn{1}{c}{4} & $15.40142910$ & $15.41546811$ &  &  \\
\multicolumn{1}{c}{8} & 15.41195836\\
\end{tabular}
	\bigbreak
\item The transformations can be computed as
	\bigbreak
$x=\dfrac{(0.5+0)+(0.5-0) x_{d}}{2}=0.25+0.25 x_{d} \quad d x=\dfrac{0.5-0}{2} d x_{d}=0.25 d x_{d}$
	\bigbreak
These can be substituted to yield
	\bigbreak
$\displaystyle I=\int_{-1}^{1}\left[10 e^{-\left(0.25+0.25 x_{d}\right)} \sin 2 \pi\left(0.25+0.25 x_{d}\right)\right]^{2} 0.25 d x_{d}$
	\bigbreak
For the two-point application, the transformed function can be evaluated using the values from Table 17.1
	\bigbreak
$I=f(-0.577350269)+f(0.577350269)=7.684096+4.313728=11.99782$
	\bigbreak
For the three-point application, the transformed function can be evaluated using the values from Table 17.1
	\bigbreak
$I=0.5555556f(-0.774596669)+0.8888889f(0)+0.5555556f(0.774596669)$
	\bigbreak
$=0.5555556(1.237449)+0.8888889(15.16327)+0.5555556(2.684915)=15.65755$
	\bigbreak
\item
\begin{lstlisting}[numbers=none]
>> format long
>> I = quad(inline('(10*exp(-x).*sin(2*pi*x)).^2'),0,0.5)


I =
	15.41260804934509 
\end{lstlisting}
\end{enumerate}



\section{}
The integral to be evaluated is
	\bigbreak
$\displaystyle I=\int_{0}^{0.75} 10\left(1-\dfrac{r}{0.75}\right)^{1 / 7} 2 \pi r d r$
	\bigbreak
\begin{enumerate}[label=\bfseries(\alph*)]
\item The tableau depicting the implementation of Romberg integration to $\varepsilon_{\mathrm{s}}=0.1 \%$ is
	\bigbreak
\begin{tabular}{lrrrr}
\multicolumn{1}{c}{iteration $\rightarrow$} & 1 & 2 & 3 & 4 \\
$\varepsilon_{a} \rightarrow$ &  &25.0000$\%$&1.0725$\%$&0.098313$\%$\\
\multicolumn{1}{c}{1}&0.00000000&10.67030554&12.88063803&13.74550712\\
\multicolumn{1}{c}{2}&8.00272915&12.74249225&13.73199355\\
\multicolumn{1}{c}{4}&11.55755148&13.67014971\\
\multicolumn{1}{c}{8}&13.14200015\\
\end{tabular}
	\bigbreak
\item The transformations can be computed as
	\bigbreak
$x=\dfrac{(0.75+0)+(0.75-0) x_{d}}{2}=0.375+0.375 x_{d} \quad d x=\dfrac{0.75-0}{2} d x_{d}=0.375 d x_{d}$
	\bigbreak
These can be substituted to yield
	\bigbreak
$\displaystyle I=\int_{-1}^{1}\left[10\left(1-\dfrac{0.375+0.375 x_{d}}{0.75}\right)^{1 / 7} 2 \pi\left(0.375+0.375 x_{d}\right)\right]^{2} 0.375 d x_{d}$
	\bigbreak
For the two-point application, the transformed function can be evaluated using the values from Table 17.1
	\bigbreak
$I=f (-0.577350269) + f (0.577350269) =14.77171$
	\bigbreak
\item
\begin{lstlisting}[numbers=none]
>> format long
>> I = quad(inline('10*(1-x/0.75).^(1/7)*2*pi.*x'),0,0.75)


I =
	14.43168560836254
\end{lstlisting}
\end{enumerate}
	\bigbreak



\section{}
The integral to be evaluated is
	\bigbreak
$\displaystyle I=\int_{2}^{8}\left(9+4 \cos ^{2} 0.4 t\right)\left(5 e^{-0.5 t}+2 e^{0.15 t}\right) d t$
	\bigbreak
\begin{enumerate}[label=\bfseries(\alph*)]
\item The tableau depicting the implementation of Romberg integration to $\varepsilon_{\mathrm{s}}=0.1 \%$ is
	\bigbreak
\begin{tabular}{crrrr}
\multicolumn{1}{c}{iteration $\rightarrow$} & 1 & 2 & 3 & 4 \\
$\varepsilon_{\mathrm{a}} \rightarrow$ &  & $7.4179 \%$ & $0.1054 \%$ & $0.001212 \%$ \\
\multicolumn{1}{c}{1} & $411.26095167$ & $317.15529472$ & $322.59571622$ & $322.34570788$ \\
\multicolumn{1}{c}{2} & $340.68170896$ & $322.25568988$ & $322.34961426$ &  \\
\multicolumn{1}{c}{4} & $326.86219465$ & $322.34374398$ &  &  \\
\multicolumn{1}{c}{8} & $323.47335665$ &  &  &  \\
\end{tabular}
	\bigbreak
\item
\begin{lstlisting}[numbers=none]
>> format long
>> y = inline('(9+4*cos(0.4*x).^2).*(5*exp(-0.5*x)+2*exp(0.15*x))')
>> I = quadl(y,2,8)


I =
	3.223483672542467e+002 
\end{lstlisting}
\end{enumerate}
	\bigbreak



\section{}
\begin{enumerate}[label=\bfseries(\alph*)]
\item The integral can be evaluated analytically as,
	\bigbreak
$\displaystyle\int_{-2}^{2}\left[\dfrac{x^{3}}{3}-3 y^{2} x+y^{3} \dfrac{x^{2}}{2}\right]_{0}^{4} d y$
	\bigbreak
$\displaystyle\int_{-2}^{2} \dfrac{(4)^{3}}{3}-3 y^{2}(4)+y^{3} \dfrac{(4)^{2}}{2} d y$
	\bigbreak
$\displaystyle\int_{-2}^{2} 21.33333-12 y^{2}+8 y^{3} d y$
	\bigbreak
$\displaystyle\left[21.33333 y-4 y^{3}+2 y^{4}\right]_{-2}^{2}$
	\bigbreak
$21.33333(2)-4(2)^{3}+2(2)^{4}-21.33333(-2)+4(-2)^{3}-2(-2)^{4}=21.33333$
	\bigbreak
\item  The operation of the dblquad function can be understood by invoking help,
	\bigbreak
\begin{lstlisting}[numbers=none]
>> help dblquad
\end{lstlisting}
	\bigbreak
A session to use the function to perform the double integral can be implemented as,
	\bigbreak
\begin{lstlisting}[numbers=none]
>> dblquad(inline('x.^2-3*y.^2+x*y.^3'),0,4,-2,2)


ans =
	21.3333 
\end{lstlisting}
\end{enumerate}
\end{document}